%\RequirePackage{kvoptions-patch}
\documentclass[a4paper, twoside, bibliography=totoc, headsepline, cleardoublepage=empty, parskip=half, draft=false]{scrbook}

\let\ifdeutsch\iffalse
\let\ifenglish\iftrue

\input{config}


\usepackage[
  title={This is a Very Long Title it is Filling Two Lines}, % Do not forget to capitalize your title correctly, you may use the following page to help you: https://capitalizemytitle.com/
  author={Sigmund Freud},
  orcid=0000-0000-0000-0000, % get your own ORCID via https://orcid.org/
  email={sigmund.freud@ifi.lmu.de},
  type=bachelor,
  institute={Institut für Informatik},
  course={Medieninformatik},
  examiner={Prof.\ Dr.\ Uwe Fessor},
  supervisor={Dipl.-Inf.\ Roman Tiker,\\Dipl.-Inf.\ Laura Stern,\\Otto Normalverbraucher,\ M.Sc.},
  startdate={July 5, 2018},
  enddate={January 5, 2019},
  copyright=ccbysa, % ccbysa, ccbynosa, cc0, nda, none
  language=english,
  aids=true, % Spefici if you used any aids in the crated of the thesis, if yes [true] if no adis were used [false]
  % Next, enter which tools were used and how according to the following taxonomy (https://ceur-ws.org/GenAI/Taxonomy.html). If no aid was used, the next three values can be ignored.
  aidstools={Gammerly, ChatGPT}, 
  adiscontributions={Grammar and spelling check, Paraphrase and reword},
  aidsextracontributions={Dall-E 2 was used to generate Figure X. }
]{lmu-thesis-cover}

\input{acronyms}

\makenoidxglossaries

\begin{document}

% DE: wird fuer Tabellen benötigt (z.B. >{centering\RBS}p{2.5cm} erzeugt einen zentrierten 2,5cm breiten Absatz in einer Tabelle
\newcommand{\RBS}{\let\\=\tabularnewline}

% EN: To avoid issues with Springer's \mathplus
%     See also http://tex.stackexchange.com/q/212644/9075
\providecommand\mathplus{+}

% EN: from hmks makros.tex - \indexify
\newcommand{\toindex}[1]{\index{#1}#1}

% DE: Tipp aus "The Comprehensive LaTeX Symbol List"
\newcommand{\dotcup}{\ensuremath{\,\mathaccent\cdot\cup\,}}

% DE: Anstatt $|x|$ $\abs{x}$ verwenden.
%     Die Betragsstriche skalieren automatisch, falls "x" etwas größer sein sollte...
\newcommand{\abs}[1]{\left\lvert#1\right\rvert}

% EN: For the algorithmic package
\newcommand{\commentchar}{\ensuremath{/\mkern-4mu/}}
\algrenewcommand{\algorithmiccomment}[1]{\hfill $\commentchar$ #1}

% DE: Seitengrößen - Gegen Schusterjungen und Hurenkinder...
\newcommand{\largepage}{\enlargethispage{\baselineskip}}
\newcommand{\shortpage}{\enlargethispage{-\baselineskip}}

\newcommand{\initialism}[1]{%
  \ifdeutsch%
    \textsc{#1}\xspace%
  \else%
    \textlcc{#1}\xspace%
  \fi%
}

\pagenumbering{arabic}
\Coverpage
\Copyright
%Eigener Seitenstil fuer die Kurzfassung und das Inhaltsverzeichnis
\deftriplepagestyle{preamble}{}{}{}{}{}{\pagemark}
%Doku zu deftriplepagestyle: scrguide.pdf
\pagestyle{preamble}
\renewcommand*{\chapterpagestyle}{preamble}



%Kurzfassung / abstract
%auch im Stil vom Inhaltsverzeichnis
\section*{Kurzfassung}

\todo{Short summary of the thesis. Here, the following questions should be answered:}
\todo{What is the specific problem addressed?}
\todo{What have you done?}
\todo{What did you find out?}
\todo{What are the implications on a larger scale?}
\todo{Should be around 0.5 pages. Not longer than 1 page.}

\cleardoublepage

\section*{Abstract}

\todo{Short summary of the thesis. Here, the following questions should be answered:}
\todo{What is the specific problem addressed?}
\todo{What have you done?}
\todo{What did you find out?}
\todo{What are the implications on a larger scale?}
\todo{Should be around 0.5 pages. Not longer than 1 page.}

\cleardoublepage


% BEGIN: Verzeichnisse

\iftex4ht
\else
  \microtypesetup{protrusion=false}
\fi

%%%
% Literaturverzeichnis ins TOC mit aufnehmen, aber nur wenn nichts anderes mehr hilft!
% \addcontentsline{toc}{chapter}{Literaturverzeichnis}
%
% oder zB
%\addcontentsline{toc}{section}{Abkürzungsverzeichnis}
%
%%%

%Produce table of contents
%
%In case you have trouble with headings reaching into the page numbers, enable the following three lines.
%Hint by http://golatex.de/inhaltsverzeichnis-schreibt-ueber-rand-t3106.html
%
%\makeatletter
%\renewcommand{\@pnumwidth}{2em}
%\makeatother
%
\tableofcontents

% Bei einem ungünstigen Seitenumbruch im Inhaltsverzeichnis, kann dieser mit
% \addtocontents{toc}{\protect\newpage}
% an der passenden Stelle im Fließtext erzwungen werden.

\listoffigures
\listoftables

% Control List of Listings
\let\iflistings\iffalse
%Wird nur bei Verwendung von der lstlisting-Umgebung mit dem "caption"-Parameter benoetigt
%\lstlistoflistings
%ansonsten:
\iflistings
  \ifdeutsch
    \listof{Listing}{Verzeichnis der Listings}
  \else
    \listof{Listing}{List of Listings}
  \fi
\fi

% Control List of Algorithms
\let\ifalgorithms\iffalse
\ifalgorithms
  %mittels \newfloat wurde die Algorithmus-Gleitumgebung definiert.
  %Mit folgendem Befehl werden alle floats dieses Typs ausgegeben
  \ifdeutsch
    \listof{Algorithmus}{Verzeichnis der Algorithmen}
  \else
    \listof{Algorithmus}{List of Algorithms}
  \fi
  %\listofalgorithms %Ist nur für Algorithmen, die mittels \begin{algorithm} umschlossen werden, nötig
\fi

% Control Glossary
\let\ifglossary\iftrue
\ifglossary
    \ifdeutsch
        \printnoidxglossary[type=\acronymtype, title=Abkürzungsverzeichnis]
    \else
        \printnoidxglossary[type=\acronymtype, title=List of Acronyms]
    \fi
\fi

\iftex4ht
\else
  %Optischen Randausgleich und Grauwertkorrektur wieder aktivieren
  \microtypesetup{protrusion=true}
\fi
% END: Verzeichnisse


% Headline and footline
\renewcommand*{\chapterpagestyle}{scrplain}
\pagestyle{scrheadings}
\pagestyle{scrheadings}
\ihead[]{}
\chead[]{}
\ohead[]{\headmark}
\cfoot[]{}
\ofoot[\usekomafont{pagenumber}\thepage]{\usekomafont{pagenumber}\thepage}
\ifoot[]{}


%% vv  scroll down for content  vv %%
%%%%%%%%%%%%%%%%%%%%%%%%%%%%%%%%%%%%%%%%%%%%%%%%%%%%%%%%%%%%%%%%%%%%%%%%%%%%%%
%
% Main content starts here
%
%%%%%%%%%%%%%%%%%%%%%%%%%%%%%%%%%%%%%%%%%%%%%%%%%%%%%%%%%%%%%%%%%%%%%%%%%%%%%%


\chapter{Introduction}
\label{sec:introduction}

This is a typical human-computer interaction thesis structure for an introduction which is structured in four paragraphs as follows:
% First Paragraph
% CORE MESSAGE OF THIS PARAGRAPH:
\todo{P1.1. What is the large scope of the problem?}
\todo{P1.2. What is the specific problem?}

% Second Paragraph
% CORE MESSAGE OF THIS PARAGRAPH:
\todo{P2.1. The second paragraph should be about what have others been doing}
\todo{P2.2. Why is the problem important? Why was this work carried out?}

% Third Paragraph
% CORE MESSAGE OF THIS PARAGRAPH:
\todo{P3.1. What have you done?}
\todo{P3.2. What is new about your work?}

% Fourth paragraph
% CORE MESSAGE OF THIS PARAGRAPH:
\todo{P4.1. What did you find out? What are the concrete results?}
\todo{P4.2. What are the implications? What does this mean for the bigger picture?}

LaTeX hints are provided in \autoref{chap:latexhints}.

\chapter{Related Work}

Describe relevant scientific literature related to your work.

\chapter{Study Design}

\section{Apparatus}

\section{Procedure}

\section{Measurements}

\section{Participants}

\chapter{Results}

\chapter{Discussion}

\chapter{Conclusion}
\label{sec:conclusion}

\todo{Outlook}


%TC:ignore
\printbibliography

All links were last followed on \today{}.


\appendix
\input{latexhints/latexhints-english}

\Aids

\Affirmation
%TC:endignore
\end{document}